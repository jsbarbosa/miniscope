% !TeX spellcheck = es_ANY
\chapter{Conclusión}\label{ch:conclusion}
Como conclusión de todo el proceso desarrollado se puede entonces asegurar que se construyó un stage de microscopia capaz de hacer barridos en tres dimensiones de una muestra determinada de manera constante y teniendo en cuenta consideraciones de resolución según la etapa de detección a usar. Adicionalmente, el código mediante el cual se pone ene ejecución el proceso considera la toma de tres puntos distribuidos en el área de interés para calcular una ecuación del plano, y por lo tanto tener en cuenta cualquier clase de desnivel en el plano de muestra. Como importante característica se puede remarcar que el stage funciona de tal manera que las medidas se pueden tomar con una velocidad que permita que se tome punto por punto de manera cuidadosa o que se haga un recorrido continuo en las tres dimensiones para abarcar una muestra de tamaño superior en una sola ejecución. Se espera a futuro poder generar una mayor cantidad de pruebas que permitan validar su funcionamiento, mejorar la velocidad de barrido probando una denominación de motor diferente, y como aporte más importante se espera hacer un acople exitoso de la parte mecánica con una etapa de detección con el fin de obtener resultados ópticos y tener un microscopio de bajo costo funcionando en su totalidad. Adicionalmente el proyecto cuenta con una librería para Python, la cual es de acceso libre y puede ser modificada por cualquiera interesado en la misma. 
  \begin{center}
    Juan Barbosa \\
    \href{mailto: js.barbosa10@uniandes.edu.co}{js.barbosa10@uniandes.edu.co}\\
    \href{https://github.com/jsbarbosa}{https://github.com/jsbarbosa}\\
    Universidad de los Andes\\
	Bogotá, Colombia
  \end{center}