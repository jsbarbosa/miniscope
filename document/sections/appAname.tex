\chapter{Código}\label{ch:userside}
El código fuente del projecto, tanto para el uso del usuario para la obtención de datos, como para la manipulación del microcontrolador al interior del mismo, se encuentra de manera libre en GitHub \footnote{\url{https://github.com/jsbarbosa/miniscope}}. El código es libre, cualquier persona lo puede inspeccionar, modificar y mejorar.

\section{Librería Python (mauscope)}
Disponible en PyPI \footnote{\url{https://pypi.python.org/pypi/mauscope}}(Python Package Index), la instalación se puede llevar a cabo usando el mánager de paquetes de Python (pip) de la siguiente forma:
\begin{lstlisting}
	pip install mauscope
\end{lstlisting}

También es posible descargar la versión de desarrollo desde GitHub seguido de su instalación:
\begin{lstlisting}
	python setup.py install
\end{lstlisting}

\subsection{constants.py}
	\lstinputlisting{../mauscope/constants.py}
\subsection{core.py}
	\lstinputlisting{../mauscope/core.py}
\subsection{commandLine.py}
	\lstinputlisting{../mauscope/commandLine.py}
\subsection{plane.py}
	\lstinputlisting{../mauscope/plane.py}


\section{Ejemplos}

\section{Internal C}\label{ch:internal}
\subsection{motor.c}
\lstinputlisting[language=C]{../microcontroller/motor.c}