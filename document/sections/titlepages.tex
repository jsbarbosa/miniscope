% !TeX spellcheck = es_ANY
\pdfbookmark[0]{Spanish title page}{label:titlepage_es}
\aautitlepage{%
  \spanishprojectinfo{
    Diseño y construcción de un stage de translación en $x$, $y$ y $z$ automatizado.
  }{%
    Scientific Theme %theme
  }{%
    Segundo periodo 2017
  }{%
    XXX % project group
  }{%
   Juan Barbosa
  }{%
    %list of supervisors
    Manu Forero Shelton, PhD
  }{%
    2 % number of printed copies
  }{%
    Diciembre 3, 2017 % date of completion
  }%
}{%department and address
  \textbf{Universidad de los Andes}\\
  Departamento de Física\\
}{% the abstract
  Con el propósito de funcionar con detectores tales como microscopios ópticos de transmisión, fluorescencia o de reflexión, se construye un stage mecánico automatizado capaz de ser controlado desde el computador, con un área de barrido de 13 cm$^2$, y pasos mínimos de 1.5 $\mu$m en todas las direcciones. La velocidad mínima es de 83 $\mu$m/s, y la máxima de 176 $\mu$m/s en modo continuo. El sistema cuenta además con la posibilidad de ingresar tres puntos en foco, para obtener la ecuación de un plano y de esta forma optimizar el enfoque a lo largo del barrido.
  }
